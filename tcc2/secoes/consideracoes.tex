%!TEX root = ../main.tex

\chapter{Considerações Finais} \label{cap:consideracoes}

%% Problema, tarefa, o uso de Redes Neurais Convolucionais para abordar esta tarefa e a separação dos dados promovendo uma proximidade com o cenário relatístico
O objetivo desta proposta de trabalho de conclusão de curso consistiu em endereçar o problema de autenticação de assinaturas manuscritas considerando a perspectiva de Aprendizado de Máquina utilizando Redes Neurais Convolucionais. Para isto, foi selecionado um conjunto de dados contendo assinaturas forjadas e genuínas, o qual foi preparado para a tarefa de interesse, contendo $27.962$ exemplos. Estes dados foram particionados em conjuntos de treino, teste e validação considerando duas abordagens distintas. Na primeira delas, chamada de abordagem A, houve a ressalva de que apenas falsificações inéditas compuseram o conjunto de teste. Na abordagem B, por outro lado, definiu-se que apenas assinaturas de autores inéditos estariam presentes no conjunto de teste. Ambas as abordagens foram concepcionadas visando aproximar a avaliação de cenários realísticos. Estas bases de dados foram então utilizadas para treinamento e teste de arquiteturas de redes neurais convolucionais bem estabelecidas na literatura.

Dentre as arquiteturas escolhidas para fazer parte deste trabalho estão a LeNet, AlexNet, MobileNet, ShuffleNet, SqueezeNet, VGG-16 e Inception-V3, as quais algumas destas passaram por uma busca em \emph{grid} que combinou vários valores de hiperparâmetros e as diferentes abordagens, culminando no treinamento e teste de um total de $222$ modelos. Dentre estes, aquele que resultou em um melhor desempenho, considerando apenas a abordagem A, é caracterizado pela arquitetura LeNet, utilizando o otimizador RMSProp, possui \emph{patience} igual a 5 e utiliza função de ativação \emph{Leaky} ReLU. Este modelo obteve uma acurácia de $98.65\%$ e um valor de \emph{F-score} igual a $0.9755$. Se considerada apenas a habilidade deste modelo em identificar falsificações, ignorando-se os resultados obtidos para assinaturas autênticas, tem-se um \emph{F-Score} igual a $0.9915$ para esta habilidade\footnote{Calculou-se este \emph{F-Score} tomando o valor $6016$ como sendo de verdadeiros positivos e o valor $103$ como sendo de falsos negativos.}. Enquanto que para a abordagem B, o melhor modelo obtido, é identificado pela arquitetura MobileNet, utiliza-se do otimizador Adam, possui um valor de \emph{patience} igual a 15 e a função de ativação ReLU. Obteve uma acurácia de $88.56\%$, \emph{F-Score} de $86.58\%$ e um EER de $9.94\%$.


Além do bom desempenho obtido, ressalta-se que as arquiteturas associadas a estes modelos são CNNs que possuem poucos parâmetros quando comparadas a outras avalidadas neste trabalho, o que agrega um valor ainda maior aos resultados obtidos em virtude do menor esforço computacional para realização de previsões e menor espaço em disco para armazenamento. 

Considerando o bom desempenho conquistado nesta tarefa e demonstrando a adequação dos modelos para o que foi proposto, tem-se em mente, em uma próxima etapa deste trabalho... \todo{Completar com planos futuros (busca em grid em arquiteturas já exploradas, redes siamesas, mais dados, etc...)}

O problema em questão é significativo do ponto de vista prático pois pode colaborar, por exemplo, para a autenticação de documentos de maneira automática e confiável diminuindo os recursos humanos especializados para este fim. Do ponto de vista do bacharel em Engenharia de Computação que desenvolveu este trabalho, construir uma solução para este problema foi a oportunidade de pôr em prática diversos conceitos aprendidos ao longo do curso, principalmente aqueles presentes nas disciplinas de Inteligência Artificial, \emph{Machine Learning}, Redes Neurais Artificiais, Linguagem de Programação, Sinais e Sistemas e Processamento Digital de Imagens.
