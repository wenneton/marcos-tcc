%!TEX root = ../main.tex

\chapter{Considerações Finais} \label{cap:consideracoes}

%% Problema, tarefa, o uso de Redes Neurais Convolucionais para abordar esta tarefa e a separação dos dados promovendo uma proximidade com o cenário relatístico
O objetivo desta proposta de trabalho de conclusão de curso consiste em endereçar o problema de autenticação de assinaturas manuscritas considerando a perspectiva de Aprendizado de Máquina utilizando Redes Neurais Convolucionais. Para isto, foi selecionado um conjunto de dados contendo assinaturas forjadas e genuínas, o qual foi preparado para a tarefa de interesse, contendo $27.962$ exemplos. Estes dados foram particionados em conjuntos de treino, teste e validação com a ressalva de que apenas falsificações inéditas compuseram o conjunto de teste, visando aproximar a avaliação de cenários realísticos. Esta base se de dados foi então utilizada para treinamento e teste de arquiteturas de redes neurais convolucionais bem estabelecidas na literatura.

Dentre as arquiteturas escolhidas, na primeira parte deste trabalho foram consideradas a LeNet e a AlexNet, as quais passaram por uma busca em \emph{grid} que combinou vários valores de hiperparâmetros, culminando no treinamento e teste de um total de $72$ modelos. Dentre estes, aquele que resultou em um melhor desempenho, é caracterizado pela arquitetura LeNet, utiliza o otimizador RMSProp, possui \emph{patience} igual a 5 e utiliza função de ativação \emph{Leaky} ReLU. Este modelo obteve uma acurácia de $98.65\%$ e um valor de \emph{F-score} igual a $0.9755$. Se considerada apenas a habilidade deste modelo em identificar falsificações, ignorando-se os resultados obtidos para assinaturas autênticas, tem-se um \emph{F-Score} igual a $0.9915$ para esta habilidade\footnote{Calculou-se este \emph{F-Score} tomando o valor $6016$ como sendo de verdadeiros positivos e o valor $103$ como sendo de falsos negativos.}. Além do bom desempenho obtido, ressalta-se que esta arquitetura é a que possui menos parâmetros dentre as avalidadas até o momento, o que agrega um valor ainda maior aos resultados obtidos em virtude do menor esforço computacional para realização de previsões e menor espaço em disco para armazenamento.

Considerando o bom desempenho já conquistado nesta tarefa e demonstrando a adequação dos modelos para o que foi proposto, tem-se em mente, na próxima etapa deste trabalho, prosseguir primeiramente com topologias que possuam menos parâmetros, em especial, as arquiteturas SqueezeNet e MobileNet. Após isto, será considerada também a avaliação de redes mais profundas, como a VGG-16 e a Inception, possivelmente utilizando técnicas de \emph{Data Augmentation}, se necessário, para contornar as limitações relativas ao tamanho do conjunto de dados disponível.

O problema em questão é significativo do ponto de vista prático pois pode colaborar, por exemplo, para a autenticação de documentos de maneira automática e confiável diminuindo os recursos humanos especializados para este fim. Do ponto de vista do bacharel em Engenharia de Computação que desenvolve este trabalho, construir uma solução para este problema foi a oportunidade de pôr em prática diversos conceitos aprendidos ao longo do curso, principalmente aqueles presentes nas disciplinas de Inteligência Artificial, \emph{Machine Learning}, Redes Neurais Artificiais, Linguagem de Programação, Sinais e Sistemas e Processamento Digital de Imagens.
