%!TEX root = ../main.tex

\chapter{Trabalhos Relacionados} \label{cap:trab-rel}

Na literatura, existem uma grande quantidade de soluções elaboradas para resolver o problema de autenticação de assinaturas manuscritas, porém os resultados mais comparáveis aos encontrados no desenvolvimento deste trabalho podem ser verificados a partir da \emph{Signature Verification Competition} ocorrida em 2009 (SigComp2009), na qual a tarefa de aprendizado abordada se assemelha à aqui considerada. Na referida competição, os participantes foram instruídos a submeterem um sistema que, ao receber uma assinatura genuína de um indivíduo como referência e uma outra assinatura para comparação, deveria retornar um grau de similaridade entre as assinaturas e um valor binário de decisão que definia a autenticidade da assinatura em questão \cite{icdar2009}. Com o intuito de analisar o desempenho dos sistemas submetidos, a organização da competição decidiu adotar a métrica de \emph{equal error rate} (EER), frequentemente utilizada na avaliação de sistemas biométricos. Esta métrica identifica um ponto de equilíbrio entre as taxas de falsa aceitação e falsa rejeição e, deste modo, quanto mais baixo for o seu valor, melhor é a qualidade do sistema biométrico analisado \cite{capsi}.

Na guia da SigComp2009 que buscava analisar sistemas que verifivam assinaturas \emph{offline}, houve a participação de oito competidores. Dentre estes, o melhor sistema verificador obteve um EER de $9.15\%$ e foi construído através de uma única abordagem que se baseava na informação de cores das imagens. O segundo melhor modelo, com um EER de $15.5\%$, foi obtido através de redes neurais artificiais que visavam encontrar o conjunto de características ideal das imagens para a classificação das assinaturas, porém os seus autores decidiram manter anônimas quaisquer outras informações a respeito da construção deste sistema \cite{icdar2009, volker}.

O trabalho de Ribeiro et al. emprega técnicas de DL para a identificação \emph{offline} de assinaturas manuscritas em um \emph{dataset} disponibilizado pelo \emph{Grupo de Procesado Digital de Senales} (GPDS). Este trabalho consiste, primeiramente, no uso de \emph{K-means} e índices de frequência obtidos através das transformadas discretas de Fourier, de cosseno e de wavelet para a extração de características das assinaturas que, em um segundo passo, foram fornecidas à Maquinas de Vetores de Suporte (SVMs) com o intuito de coletar métricas para análise posterior dos modelos obtidos. A abordagem deste trabalho considerou a criação de um modelo híbrido que constitui-se de uma dupla validação, nas quais na primeira destas, o modelo deve identificar o proprietário da assinatura em questão e, subsequentemente, determinar a sua autenticidade. Dentre as métricas coletadas, as principais foram as taxas de falsa aceitação e falsa rejeição, a acurácia e o \emph{F-score}, na qual a última destas obteve um valor de $0.8615$ no melhor modelo produzido. Em um terceiro passo do trabalho, uma pequena parte dos dados utilizados anteriormente foi disponibilizada à uma \emph{Restricted Boltzmann Machine}, visando apenas a demonstração visual dos pesos obtidos por este de tipo de rede profunda, não havendo a existência de testes dessas características quanto à classificação da autenticidade \cite{ribeiro2011}.

Mais recentemente, Hafemann et al. propuseram o aprendizado de características de assinaturas manuscritas \emph{offline} utilizando redes neurais convolucionais em conjunto com SVMs. O conjunto de dados utilizado para o treinamento dos modelos foi também disponibilizado pelo GPDS. As abordagens para a solução do problema foram diversas, porém, a que mais se destacou foi aquela na qual os modelos gerados classicavam a autenticidade da assinatura de forma independente dos autores das mesmas. O melhor dentre estes modelos obteve um EER de $1.72\%$, conseguindo superar o estado da arte até aquele momento \cite{hafemann2016}.

Levando em conta o estado da arte, as diferentes estratégias para abordar o problema e ainda os poucos trabalhos envolvendo redes neurais convolucionais, verifica-se a importância de perseguir esta perspectiva e colaborar com resultados que visem avaliar o potencial de tais modelos no cenário em questão. 

\todo{Inserir um parágrafo sobre State of the Art}