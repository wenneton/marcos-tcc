\ \ \\[2cm]
Este trabalho apresenta uma proposta para a verificação da autenticidade de assinaturas manuscritas feitas por indivíduos utilizando conceitos de \emph{Deep Learning} por meio da aplicação de redes neurais convolucionais. Neste contexto, o problema da verificação de autenticidade de assinaturas manuscritas é tratado como um problema de classificação binária, em que os exemplos dispostos como entrada para os modelos consistiam em imagens compostas por duas assinaturas, uma de referência e outra para inferência. Para a tarefa proposta foram conduzidas abordagens de treinamento e teste de diversos modelos de redes neurais convolucionais, sujeitas à variações de hiperparâmetros. Os resultados mostraram que os melhores modelos para esta tarefa foram baseados na arquitetura LeNet, para a primeira abordagem, e MobileNet, na segunda abordagem. Estes modelos obtiveram um valor de $88.56\%$ e $98.65\%$ de acurácia, respectivamente, evidenciando o potencial de adequação dos modelos propostos para tarefas dessa natureza.

\noindent \textbf{Palavras-chave}: Verificação de autenticidade, Redes Neurais Convolucionais, \emph{Deep Learning}
