\ \ \\[2cm]
Este trabalho apresenta uma proposta para a verificação da autenticidade de assinaturas manuscritas feitas por indivíduos utilizando conceitos de \emph{Deep Learning} através da aplicação de redes neurais convolucionais. Neste contexto, o problema da verificação de autenticidade de assinaturas manuscritas é tratado como um problema de classificação binária. Para caracterizá-la, são conduzidas duas abordagens de treinamento e teste de diversos modelos de redes neurais convolucionais, sujeitadas à variações de hiperparâmetros. Os resultados mostraram que os melhores modelos para esta tarefa foram baseados na arquitetura LeNet, para a primeira abordagem, e MobileNet, na segunda abordagem. Estes modelos obtiveram um valor de $88.56\%$ e $98.65\%$ de acurácia, respectivamente. Os exemplos dispostos às entradas destes modelos foram imagens compostas por duas assinaturas, uma de referência e outra para inferência. Como trabalhos futuros...\todo{Inserir o que fazer em trabalhos futuros aqui}

\textbf{Palavras-chave}: Verificação de autenticidade, Redes Neurais Convolucionais, \emph{Deep Learning}
