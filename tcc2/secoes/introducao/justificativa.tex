%!TEX root = ../../main.tex

Apesar da capacidade tecnológica atual e da proposição crescente de diversas características e métodos para autenticação biométrica, as assinaturas manuscritas ainda possuem um papel importante em nossa sociedade, estando presentes na ampla maioria dos documentos oficiais do País e servindo também para comprovação de diversas transações financeiras. Outro aspecto que ressalta a importância deste trabalho consiste na possibilidade da adoção dos modelos elaborados na verificação de autenticidade de documentos históricos ou artísticos, colaborando também na diminuição de fraudes nestes âmbitos. Assim, é importante a contínua proposição, melhoria e avaliação de métodos para este fim, com vistas a aumentar a eficiência e diminuir as eventuais vulnerabilidades.

Considerando que as técnicas de \emph{Deep Learning} ainda são emergentes, é importante propor trabalhos que possam ajudar a verificar a adequação destas soluções, indicando vantagens e limitações, bem como comparações com o estado da arte.

No mais, do ponto de vista do bacharel em Engenharia de Computação em formação, a proposta de trabalho de conclusão de curso corrobora para a prática de conceitos, tecnologias e métodos de uma área emergente do Aprendizado de Máquina que é o \emph{Deep Learning}. Por fim, deve-se mencionar a importância da realização deste trabalho com vistas a colaborar com as atividades desenvolvidas pelo \emph{Laboratório de Sistemas Inteligentes} (LSI), uma iniciativa do \emph{Grupo de Pesquisas em Sistemas Inteligentes} da Escola Superior de Tecnologia (EST) da Universidade do Estado do Amazonas (UEA).
