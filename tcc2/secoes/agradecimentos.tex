Agradeço primeiramente a Deus, com Ele ao meu lado eu sou forte, confiante e não tenho medo de nada e ninguém.

Agradeço o grande e total apoio de minha mãe, Gessy Vieira dos Santos. Sem seu apoio, sustento, conselhos e incentivos eu, definitivamente, não conseguiria alcançar todas as coisas nas quais deposito o meu orgulho hoje. 

Agradeço também aos outros membros da minha família, minha irmã Denize, minha sobrinha Emanuella, meu pai Walter, minha avó Antônia, minha tia Marionete e minha prima Jamille, que sempre me fizeram acreditar no meu potencial e me ajudaram a ser uma pessoa melhor.

Agradeço à incrível Profa. Dra. Elloá B. Guedes, por ter confiado na minha capacidade e aceitado me orientar neste trabalho. Esta professora serviu como uma grande base de inspiração e aprendizado. Agradeço pelos conhecimentos aprendidos durante o desenvolvimento deste trabalho e  durante todas disciplinas ministradas em que tive o prazer de participar.

Aos meus colegas de curso que estiveram comigo durante toda essa caminhada. Agradeço à Gillane, Karen e Syra que me ajudaram nos primórdios da minha experiência acadêmica. Agradeço de coração aos colegas do DNCG que acabaram se tornando meus grandes amigos. Nicoli, Rafaela, Giovana, Janderson, Luiz, Lucas, Leo, Miranda e Emanuel, sem os nossos momentos de descontração teria sido muito mais difícil aguentar toda a pressão.

Agradeço ao Instituto de Desenvolvimento Tecnológico (INDT) por me proporcionar uma experiência profissional digna e, principalmente, por permitir a associação de meus estudos às minhas atividades ali desenvolvidas. Agradeço o apoio moral e a compreensão de meus colegas de trabalho. Um agradecimento especial a todos os membros do Lanchus, com eles meus dias se tornam muito mais divertidos.

Agradeço a todos os professores do Núcleo de Computação com os quais tive a oportunidade de aprender, que os ensinamentos absorvidos por mim através deles, possam ser passados para vários outros grandes talentos. Agradeço à Universidade do Estado do Amazonas, seus servidores e alunos. Agradeço ao Governo do Estado do Amazonas por garantir subsídios à esta instituição de ensino e, desta maneira, auxiliar o crescimento da nossa sociedade.

Agradeço à Fundação de Amparo à Pesquisa do Estado do Amazonas (FAPEAM) que, por meio do Projeto PROTI Pesquisa 11/2017, colaborou para a consolidação da infraestrutura física e tecnológica do Laboratório de Sistemas Inteligentes da Escola Superior de Tecnologia da Universidade do Estado do Amazonas. Este trabalho de conclusão de curso é um dos produtos deste projeto, pois foi desenvolvido no referido laboratório, fez uso dos recursos computacionais ali disponíveis e foi melhorado graças às discussões e interações com o grupo de pesquisa nele sediado.