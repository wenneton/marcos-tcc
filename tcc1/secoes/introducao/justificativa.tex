%!TEX root = ../../main.tex


Apesar da capacidade tecnológica atual e da descoberta de uma numerosa quantidade de métodos de biometria, sabe-se que as assinaturas manuscritas ainda possuem papel importante na autenticação de diversos documentos. Justificando, portanto, a necessidade da busca por modelos mais eficientes para a realização da verificação dessas assinaturas.

Como mencionado anteriormente, as assinaturas manuscritas possuem a propriedade de ser pouco invasiva e têm um baixo custo de aquisição. Dessa maneira, a utilização desse método de biometria possui ainda grande relevância atualmente.

Além disso, um problema com a autentificação de identidade de documentos históricos como, por exemplo, quadros, testamentos e cartas importantes que possuem apenas assinaturas manuscritas por pertecerem a épocas onde as tecnologias biométricas não existiam, pode ser facilmente resolvido com os modelos propostos neste trabalho, desde que existam outros documentos com as assinaturas reais dos indivíduos associados aos achados históricos.

No mais, do ponto de vista do bacharel em Engenharia de Computação em formação, a proposta de trabalho de conclusão de curso corrobora para a prática de conceitos, tecnologias e métodos de uma área emergente do Aprendizado de Máquina que é o \emph{Deep Learning}. Por fim, deve-se mencionar a importância da realização deste trabalho com vistas a colaborar com as atividades desenvolvidas pelo \emph{Laboratório de Sistemas Inteligentes} (LSI), uma iniciativa do \emph{Grupo de Pesquisas em Sistemas Inteligentes} da Escola Superior de Tecnologia (EST) da Universidade do Estado do Amazonas (UEA).


%% Pq resolver esse problema é importante

% Análise de documentos históricos
% Método não invasivo de biometria
% Custo baixo da obtenção de Assinaturas
%
% Área emergente Deep learning
% Atividades desenvolvidas pelo LSI
