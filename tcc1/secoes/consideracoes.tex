%% Problema, tarefa, o uso de Redes Neurais Convolucionais para abordar esta tarefa e a separação dos dados promovendo uma proximidade com o cenário relatístico

% Contabilizar quais arquiteturas já foram treinadas
% Quantos modelos ao total
% Melhor resultado (acurácia e F-Score)
% Menos parâmetros dentre as arquiteturas avaliadas até o momento


% Considerando o bom desempenho já obtido nesta tarefa, demonstrando a adequação dos modelos para o que foi proposto, tem-se em mente na próxima etapa prosseguir primeiramente com modelos que tenham menos parâmetros, em especial, squeeze e mobile, visando avaliar se estas podem resultar melhores resultados com menos esforço computacional. Após estes resultados, será considerada também a avaliação de redes mais profundas, possivelmente utilizando técnicas de data augmentation para contornar as limitações relativas ao tamanho do conjunto de dados disponível
