
\subsection{Objetivos}

O objetivo geral deste trabalho consiste em verificar a autenticidade de assinaturas manuscritas com redes neurais convolucionais. Para alcançar esta meta, alguns objetivos específicos precisam ser consolidados, a citar:

\begin{enumerate}
  \item Realizar a fundamentação teórica acerca dos conceitos das redes neurais convolucionais;
  \item Consolidar uma base de dados representativa de assinaturas;
  \item Descrever o problema considerado como uma tarefa de Aprendizado de Máquina;
  \item Propor, treinar e testar redes neurais convolucionais para a tarefa considerada;
  \item Analisar os resultados obtidos.
\end{enumerate}

\subsection{Justificativa}

%% Pq resolver esse problema é importante

% Análise de documentos históricos

\subsection{Metodologia}

Para alcançar os objetivos propostos no escopo deste trabalho, a condução das atividades a serem realizadas obedecerá à metodologia descrita a seguir:

\begin{enumerate}
  \item Estudo dos conceitos relacionados ao Aprendizado de Máquinas, Redes Neurais Convolucionais e \emph{Deep Learning};
  \item Descrição do problema considerado como uma tarefa de Aprendizado de Máquina;
  \item Consolidação de uma base de dados representativa de assinaturas originais e forjadas;
  \item Levantamento do ferramental tecnológico para implementação das redes neurais convolucionais;
  \item Proposição de modelos de redes neurais convolucionais para o problema considerado, contemplando arquitetura, parâmetros e hiperparâmetros;
  \item Treino das redes propostas para a tarefa de aprendizado considerada;
  \item Teste das redes previamente treinadas com vistas a coleta de métricas de desempenho;
  \item Análise dos resultados e identificação dos modelos mais adequados para o problema considerado;
  \item Escrita da proposta de Trabalho de Conclusão de Curso;
  \item Defesa da proposta de Trabalho de Conclusão de Curso;
  \item Escrita do Trabalho de Conclusão de Curso; e
  \item Defesa do Trabalho de Conclusão de Curso.
\end{enumerate}

\subsection{Cronograma}

Considerando as atividades enumeradas na metodologia, a Tabela \ref{tab:cronograma} sintetiza o cronograma de execução deste trabalho.

\begin{table}
\scalefont{0.8}
\caption{Cronograma de atividades levando em consideração os dez meses (de $02/2019$ a $12/2019$) para a realização do TCC.}
\label{tab:cronograma}

\begin{center}
\begin{small}
\begin{tabular}{p{5cm}cccccccccccc}
  \toprule
  & &  &  & &  & \textbf{2019}  & &  &  &  &  & \\
                                        & \textbf{02} & \textbf{03} & \textbf{04} & \textbf{05} & \textbf{06} & \textbf{07} & \textbf{08} & \textbf{09} & \textbf{10} & \textbf{11} & \textbf{12} \\
  \midrule

  \textbf{Atividade 1}                     &      X      &      X      &      X      &           &             &             &             &             &             &             &             \\
  \textbf{Atividade 2} &             &      X      &            &             &             &             &             &             &             &             &             \\
  \textbf{Atividade 3}         &             &     X        &    X         &            &            &            &            &            &             &             &             \\
  \textbf{Atividade 4}         &             &             &    X         &            &            &            &            &            &             &             &             \\
  \textbf{Atividade 5}         &             &             &             &      X      &       X     &   X         &     X       &            &             &             &             \\
  \textbf{Atividade 6}         &             &             &             &      X      &     X       &     X       &    X        &            &             &             &             \\
  \textbf{Atividade 7}         &             &             &             &            &            &            &      X      &      X      &             &             &             \\
  \textbf{Atividade 8}         &             &             &             &            &            &            &           &            &   X          &      X       &             \\
  \textbf{Atividade 9}          &      X      &      X      &      X      &      X      &      X      &             &             &             &             &             &             \\

  \textbf{Atividade 10}          &             &             &             &             &      X      &             &             &             &             &             &             \\
  \textbf{Atividade 11}    &             &             &             &             &             &      X      &      X      &      X      &      X      &      X      &      X      \\
  \textbf{Atividade 12}     &             &             &             &             &             &             &             &             &             &             &      X      \\
  \bottomrule
\end{tabular}
\end{small}
\end{center}
\end{table}


\subsection{Organização do Documento}

(..)
