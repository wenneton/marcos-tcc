%!TEX root = ../main.tex

% Falar sobre a necessidade de autenticação de identidade pela humanidade

A verificação automática de identidade por biometria tornou-se nos últimos anos uma rotina diária na vida de várias pessoas. A biometria é definida como a utilização de características fisiológicas ou traços comportamentais para a comprovação da identidade de um indivíduo. A autenticação por biometria ganhou muita popularidade como uma alternativa confiável para sistemas baseados em segurança por chave, devido suas propriedades únicas e a capacidade quase nula de cópia, roubo ou adivinhação \cite{kholmatov}.

Dentre as biometrias que utilizam-se de características fisiológicas estão, por exemplo, as impressões digitais, o exame de fundo de retina e da palma da mão. Estas técnicas são muito mais seguras do que senhas convecionais, porém deparam-se com problemas operacionais que dificultam o seu uso em larga escala, como o grande custo com \emph{hardwares} especializados e o grau de intrusão elevado aos usuários na captura destas características \cite{[5]noheinen2002}. Por outro lado, existem as características biométricas que utilizam-se de traços comportamentais como, por exemplo, expressões faciais, gestos e assinaturas manuscritas. Dentre essas técnicas, a que mais se destaca é a autenticação de identidade através de assinaturas manuscritas, devido às suas diversas vantagens em relação às outras técnicas \cite{heinen2002}.

Uma das vantagens das assinaturas manuscritas é a segurança, pois ao contrário das senhas, mesmo que alguém conheça a assinatura de um usuário, normalmente não é possível reproduzi-la de forma trivial. Outra de suas vantagens é a familiaridade de alguns seres humanos em utilizar suas assinaturas como uma forma de autenticação em transações financeiras, e assim sentem-se mais seguros e confortáveis em relação ao seu uso \cite{heinen2004}. Além disso, este método biométrico possui ainda a característica de ser pouco invasivo e ter um baixo custo de aquisição, aumentando ainda mais a sua aceitação como método de autenticação \cite{souza2009}.

% (?) Falar sobre as assinaturas online e offline na introdução?

O maior desafio das técnicas de verificação de assinaturas é determinar se uma assinatura em questão é, de fato, escrita por quem afirma ser e se falsificações podem ser identificadas. Apesar de todas as vantagens apresentadas, a autenticação automatizada de assinaturas manuscritas não é uma tarefa trivial de se realizar, pois ao contrário das impressões digitais e do exame de fundo de retina, onde o padrão a ser reconhecido varia muito pouco, as assinaturas manuscritas apresentam uma grande variabilidade, inclusive entre diferentes assinaturas de um mesmo indivíduo \cite{heinen2002}. Na literatura já foram explorados diversos métodos com vistas a superar este desafio, nos quais bons resultados foram encontrados principalmente utilizando Máquinas de Vetores de Suporte, Modelos Escondidos de Markov, Análise do Componente Principal, \emph{Dynamic Time Warping}, dentre outros \cite{souza2009}.

O \emph{Deep Learning} (DL), às vezes chamado de aprendizado profundo, é uma subárea da Inteligência Artificial que possui grande destaque na resolução de problemas de Visão Computacional. Com base nisso, para este trabalho, utilizou-se técnicas de DL envolvendo Redes Neurais Convolucionais para a determinação de padrões existentes em imagens de assinaturas de diversos autores diferentes, a fim de criar modelos capazes de identificar falsificações em assinaturas.

% Ultimo páragrafo para relacionar as próximas seçẽos da introdução?

\subsection{Objetivos}

O objetivo geral deste trabalho consiste em verificar a autenticidade de assinaturas manuscritas com redes neurais convolucionais. Para alcançar esta meta, alguns objetivos específicos precisam ser consolidados, a citar:

\begin{enumerate}
  \item Realizar a fundamentação teórica acerca dos conceitos das redes neurais convolucionais;
  \item Consolidar uma base de dados representativa de assinaturas;
  \item Descrever o problema considerado como uma tarefa de Aprendizado de Máquina;
  \item Propor, treinar e testar redes neurais convolucionais para a tarefa considerada;
  \item Analisar os resultados obtidos.
\end{enumerate}

\subsection{Justificativa}
%!TEX root = ../main.tex

\begin{frame}{Justificativa}
	\begin{itemize}
		\item Autenticação de assinaturas manuscritas
		\begin{itemize}
			\item Devido a ampla utilização em documentos oficiais e transações
			financeiras atualmente, busca-se a melhoria e avaliação de métodos para este fim;
			\item Documentos e obras de arte históricas.
		\end{itemize}
		\bigskip
		\item Prática de conceitos, técnicas e tecnologias de uma área emergente da Computação
		\bigskip
		\item Proposta alinhada com as atividades desenvolvidas pelo \alert{Laboratório de Sistemas Inteligentes}
	\end{itemize}
\end{frame}


\subsection{Metodologia}
%!TEX root = ../main.tex

\begin{frame}{Metodologia}
	\begin{itemize}
		\item[] A condução das atividades obedece à metodologia apresentada a seguir, composta dos seguintes passos:
		\medskip
		\begin{itemize}
			\footnotesize
			\item[1.] Estudo dos conceitos relacionados à Aprendizado de Máquina, Redes Neurais Convolucionais e \emph{Deep Learning};
			\medskip
			\medskip
			\item[2.] Descrição do problema considerado como uma tarefa de Aprendizado de Máquina;
			\medskip
			\medskip
			\item[3.] Consolidação de uma base de dados representativa de assinaturas originais e forjadas;
			\medskip
			\medskip
			\item[4.] Levantamento do ferramental tecnológico para implementação das redes neurais convolucionais;
			\medskip
			\medskip
			\item[5.] Proposição de modelos de redes neurais convolucionais para o problema considerado, contemplando arquitetura, parâmetros e hiperparâmetros;
		\end{itemize}
	\end{itemize}
\end{frame}

\begin{frame}{Metodologia}
\begin{itemize}
	\item[]
	\begin{itemize}
		\footnotesize
		\item[6.] Treino das redes propostas para a tarefa de aprendizado considerada;
		\medskip
		\medskip
		\item[7.] Teste das redes previamente treinadas com vistas a coleta de métricas de desempenho;
		\medskip
		\medskip
		\item[8.] Análise dos resultados e identificação dos modelos mais adequados para o problema considerado;
		\medskip
		\medskip
		\item[9.] Escrita da proposta de Trabalho de Conclusão de Curso;
		\medskip
		\medskip
		\item[10.] Defesa da proposta de Trabalho de Conclusão de Curso;
		\medskip
		\medskip
		\item[11.] Escrita do Trabalho de Conclusão de Curso; e
		\medskip
		\medskip
		\item[12.] Defesa do Trabalho de Conclusão de Curso.
	\end{itemize}
\end{itemize}
\end{frame}


\subsection{Cronograma}
%!TEX root = ../main.tex

\begin{frame}{Cronograma }
  \ \  \\[0.1cm]
  \begin{minipage}[b]{0.9\linewidth}
\begin{table}[h!]
\scalefont{0.5}
\begin{center}
\caption{Cronograma de atividades}
\label{tab:cronograma}
\begin{small}
\begin{tabular}{p{5cm}cccccccccccc}
  \toprule
  & &  &  & &  & \textbf{2019}  & &  &  &  &  & \\
                                        & \textbf{02} & \textbf{03} & \textbf{04} & \textbf{05} & \textbf{06} & \textbf{07} & \textbf{08} & \textbf{09} & \textbf{10} & \textbf{11} & \textbf{12} \\
  \midrule

  \textbf{Atividade 1}                     &      X      &      X      &      X      &           &             &             &             &             &             &             &             \\
  \textbf{Atividade 2} &             &      X      &            &             &             &             &             &             &             &             &             \\
  \textbf{Atividade 3}         &             &     X        &    X         &            &            &            &            &            &             &             &             \\
  \textbf{Atividade 4}         &             &             &    X         &            &            &            &            &            &             &             &             \\
  \textbf{Atividade 5}         &             &             &             &      X      &       X     &   X         &     X       &            &             &             &             \\
  \textbf{Atividade 6}         &             &             &             &      X      &     X       &     X       &    X        &            &             &             &             \\
  \textbf{Atividade 7}         &             &             &             &            &            &            &      X      &      X      &             &             &             \\
  \textbf{Atividade 8}         &             &             &             &            &            &            &           &            &   X          &      X       &             \\
  \textbf{Atividade 9}          &      X      &      X      &      X      &      X      &      X      &             &             &             &             &             &             \\

  \textbf{Atividade 10}          &             &             &             &             &      X      &             &             &             &             &             &             \\
  \textbf{Atividade 11}    &             &             &             &             &             &      X      &      X      &      X      &      X      &      X      &      X      \\
  \textbf{Atividade 12}     &             &             &             &             &             &             &             &             &             &             &      X      \\
  \bottomrule
\end{tabular}
\end{small}
\end{center}
\end{table}
\end{minipage}
\end{frame}



\subsection{Organização do Documento}
%!TEX root = ../../novoIndex.tex

Para a apresentação deste trabalho de conclusão de curso, este documento encontra-se dividido nas seções a seguir. O Capítulo \ref{cap:fund-teor} relaciona os fundamentos teóricos que se fizeram necessários para a resolução do problema apresentado. No Capítulo \ref{cap:trab-rel} é descrita uma análise de trabalhos relacionados. O capítulo \ref{cap:sol-prop}, por sua vez, discorre sobre a solução proposta para a verificação da autenticidade de assinaturas manuscritas, que é seguida pelos resultados e discussões considerando estas soluções, e estão relatados no Capítulo \ref{cap:resultados}. Por fim, no Capítulo \ref{cap:consideracoes}, encontram-se as considerações finais sobre a realização deste trabalho.

