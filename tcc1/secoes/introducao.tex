%!TEX root = ../main.tex

\added{A autenticação possui importância fundamental na segurança de sistemas computacionais, pois consiste em permitir ou negar acesso à certas informações ou servicos com base na identidade associada à entidade que solicita acesso ao recurso ou em algum atributo que depende desta identidade \cite{Costa:Biometria}.}

A Biometria, em particular, tem sido uma das técnicas mais difundidas para autenticação, já sendo utilizada de forma abrangente na rotina diária da população em geral. Ela é definida como a utilização de características fisiológicas ou traços comportamentais para a comprovação da identidade de um indivíduo. A autenticação por biometria ganhou muita popularidade como uma alternativa confiável para sistemas baseados em segurança por chave, devido suas propriedades únicas e a capacidade quase nula de cópia, roubo ou adivinhação \cite{kholmatov}.

Dentre as técnicas de autenticação por Biometria, tem-se aquelas baseadas em características fisiológicas do indivíduo, as quais levam em conta, por exemplo, as impressões digitais, o exame de fundo de retina, a palma da mão e até a arcada dentária. Estas técnicas são muito seguras, mas ainda incorrem em problemas operacionais que dificultam o seu uso em larga escala, como o custo com \emph{hardware} e o grau de intrusão elevado aos usuários na captura destas características \cite{heinen2002}. \added{Existem também técnicas de autenticação biométrica que utilizam-se de traços comportamentais como, por exemplo, expressões faciais, gestos, assinaturas manuscritas e modo de andar, dentre outras, as quais são características dinâmicas e que podem variar fortemente ao longo do tempo.  Porém, apesar destes desafios, podem ter seus padrões capturados mediante experiência \cite{Costa:Biometria}.}

\added{A assinatura manuscrita, em especial, é particularmente utilizada para autenticação biométrica de identidade desde os tempos primórdios. Ela é caracterizada pela produção, de próprio punho, de uma marca referente ao nome ou rubrica do autor como uma prova de sua identidade \cite{heinen2002}. Nos sistemas biométricos de autenticação, apresenta vantagens em relação às senhas, por exemplo, por possuir reprodução não-trivial. Citam-se também os aspectos não-invasivos na sua obtenção, diferentemente, por exemplo, da análise de certas características fisiológicas, e também o baixo custo de aquisição, o que colabora para sua ampla difusão \cite{heinen2004,souza2009}.}\deleted{Dentre essas técnicas, a que mais se destaca é a autenticação de identidade através de assinaturas manuscritas, devido às suas diversas vantagens em relação às outras técnicas. Uma das vantagens das assinaturas manuscritas é a segurança, pois ao contrário das senhas, mesmo que alguém conheça a assinatura de um usuário, normalmente não é possível reproduzi-la de forma trivial. Outra de suas vantagens é a familiaridade de alguns seres humanos em utilizar suas assinaturas como uma forma de autenticação em transações financeiras, e assim sentem-se mais seguros e confortáveis em relação ao seu uso \cite{heinen2004}. Além disso, este método biométrico possui ainda a característica de ser pouco invasivo e ter um baixo custo de aquisição, aumentando ainda mais a sua aceitação como método de autenticação \cite{souza2009}.}

% (?) Falar sobre as assinaturas online e offline na introdução?

O maior desafio das técnicas de autenticação de assinaturas é determinar se uma assinatura em questão é, de fato, escrita por quem afirma ser e se falsificações podem ser identificadas. Apesar de todas as vantagens previamente mencionadas desta informação biométrica, a criação de métodos automáticos para autenticação de assinaturas manuscritas não é uma tarefa trivial, pois, ao contrário das características biométricas fisiológicas, os padrões ali existentes podem apresentar grande variabilidade para um mesmo indivíduo \cite{heinen2002}.  Levando em conta estes desafios, a literatura já contempla diversos métodos com vistas a endereçar esta tarefa, nos quais bons resultados foram encontrados principalmente utilizando Máquinas de Vetores de Suporte, Modelos Escondidos de Markov, Análise do Componente Principal, \emph{Dynamic Time Warping}, dentre outros \cite{souza2009}.

\added{Porém, com o crescente desenvolvimento de técnicas de \emph{Deep Learning} aplicadas à problemas de Visão Computacional, houve o desejo de investigar o desempenho de tais técnicas na autenticação de assinaturas, problema que está sendo considerado no escopo deste trabalho. Esta subárea do Aprendizado de Máquina, inserida no escopo da Inteligência Artificial, baseia-se na proposição de modelos que aprendem a partir da experiência, sendo muito aplicados em problemas de classificação, detecção, localização e segmentação de objetos em imagens, com muitos resultados expressivos em diversos domínios \cite{khan}. Assim, almeja-se investigar as capacidades de tais modelos, especialmente baseados no uso de Redes Neurais Convolucionais, na identificação de assinaturas autênticas e forjadas de diversos indivíduos.}

\deleted{O \emph{Deep Learning} (DL), às vezes chamado de aprendizado profundo, é uma subárea da Inteligência Artificial que possui grande destaque na resolução de problemas de Visão Computacional. Com base nisso, para este trabalho, utilizou-se técnicas de DL envolvendo Redes Neurais Convolucionais para a determinação de padrões existentes em imagens de assinaturas de diversos autores diferentes, a fim de criar modelos capazes de identificar falsificações em assinaturas.}

% Ultimo páragrafo para relacionar as próximas seçẽos da introdução?

\subsection{Objetivos}

O objetivo geral deste trabalho consiste em verificar a autenticidade de assinaturas manuscritas com redes neurais convolucionais. Para alcançar esta meta, alguns objetivos específicos precisam ser consolidados, a citar:

\begin{enumerate}
  \item Realizar a fundamentação teórica acerca dos conceitos das redes neurais convolucionais;
  \item Consolidar uma base de dados representativa de assinaturas;
  \item Descrever o problema considerado como uma tarefa de Aprendizado de Máquina;
  \item Propor, treinar e testar redes neurais convolucionais para a tarefa considerada;
  \item Analisar os resultados obtidos.
\end{enumerate}

\subsection{Justificativa}
%!TEX root = ../../main.tex


Apesar da capacidade tecnológica atual e da descoberta de uma numerosa quantidade de métodos de biometria, sabe-se que as assinaturas manuscritas ainda possuem papel importante na autenticação de diversos documentos. Justificando, portanto, a necessidade da busca por modelos mais eficientes para a realização da verificação dessas assinaturas.

Como mencionado anteriormente, as assinaturas manuscritas possuem a propriedade de ser pouco invasiva e têm um baixo custo de aquisição. Dessa maneira, a utilização desse método de biometria possui ainda grande relevância atualmente.

Além disso, um problema com a autentificação de identidade de documentos históricos como, por exemplo, quadros, testamentos e cartas importantes que possuem apenas assinaturas manuscritas por pertecerem a épocas onde as tecnologias biométricas não existiam, pode ser facilmente resolvido com os modelos propostos neste trabalho, desde que existam outros documentos com as assinaturas reais dos indivíduos associados aos achados históricos.

No mais, do ponto de vista do bacharel em Engenharia de Computação em formação, a proposta de trabalho de conclusão de curso corrobora para a prática de conceitos, tecnologias e métodos de uma área emergente do Aprendizado de Máquina que é o \emph{Deep Learning}. Por fim, deve-se mencionar a importância da realização deste trabalho com vistas a colaborar com as atividades desenvolvidas pelo \emph{Laboratório de Sistemas Inteligentes} (LSI), uma iniciativa do \emph{Grupo de Pesquisas em Sistemas Inteligentes} da Escola Superior de Tecnologia (EST) da Universidade do Estado do Amazonas (UEA).


%% Pq resolver esse problema é importante

% Análise de documentos históricos
% Método não invasivo de biometria
% Custo baixo da obtenção de Assinaturas
%
% Área emergente Deep learning
% Atividades desenvolvidas pelo LSI


\subsection{Metodologia}
%!TEX root = ../../main.tex

Para alcançar os objetivos propostos no escopo deste trabalho, a condução das atividades que foram realizadas obedeceram à metodologia descrita a seguir:

\begin{enumerate}
  \item Estudo dos conceitos relacionados ao Aprendizado de Máquinas, Redes Neurais Convolucionais e \emph{Deep Learning};
  \item Descrição do problema considerado como uma tarefa de Aprendizado de Máquina;
  \item Consolidação de uma base de dados representativa de assinaturas originais e forjadas;
  \item Levantamento do ferramental tecnológico para implementação das redes neurais convolucionais;
  \item Proposição de modelos de redes neurais convolucionais para o problema considerado, contemplando arquitetura, parâmetros e hiperparâmetros;
  \item Treino das redes propostas para a tarefa de aprendizado considerada;
  \item Teste das redes previamente treinadas com vistas a coleta de métricas de desempenho;
  \item Análise dos resultados e identificação dos modelos mais adequados para o problema considerado;
  \item Escrita da proposta de Trabalho de Conclusão de Curso;
  \item Defesa da proposta de Trabalho de Conclusão de Curso;
  \item Escrita do Trabalho de Conclusão de Curso; e
  \item Defesa do Trabalho de Conclusão de Curso.
\end{enumerate}


\subsection{Cronograma}
Considerando as atividades enumeradas na metodologia, a Tabela \ref{tab:cronograma} sintetiza o cronograma de execução deste trabalho.

\begin{table}[h!]
\scalefont{0.8}
\caption{Cronograma de atividades levando em consideração os dez meses (de $02/2019$ a $12/2019$) para a realização do TCC.}
\label{tab:cronograma}

\begin{center}
\begin{small}
\begin{tabular}{p{5cm}cccccccccccc}
  \toprule
  & &  &  & &  & \textbf{2019}  & &  &  &  &  & \\
                                        & \textbf{02} & \textbf{03} & \textbf{04} & \textbf{05} & \textbf{06} & \textbf{07} & \textbf{08} & \textbf{09} & \textbf{10} & \textbf{11} & \textbf{12} \\
  \midrule

  \textbf{Atividade 1}                     &      X      &      X      &      X      &           &             &             &             &             &             &             &             \\
  \textbf{Atividade 2} &             &      X      &            &             &             &             &             &             &             &             &             \\
  \textbf{Atividade 3}         &             &     X        &    X         &            &            &            &            &            &             &             &             \\
  \textbf{Atividade 4}         &             &             &    X         &            &            &            &            &            &             &             &             \\
  \textbf{Atividade 5}         &             &             &             &      X      &       X     &   X         &     X       &            &             &             &             \\
  \textbf{Atividade 6}         &             &             &             &      X      &     X       &     X       &    X        &            &             &             &             \\
  \textbf{Atividade 7}         &             &             &             &            &            &            &      X      &      X      &             &             &             \\
  \textbf{Atividade 8}         &             &             &             &            &            &            &           &            &   X          &      X       &             \\
  \textbf{Atividade 9}          &      X      &      X      &      X      &      X      &      X      &             &             &             &             &             &             \\

  \textbf{Atividade 10}          &             &             &             &             &      X      &             &             &             &             &             &             \\
  \textbf{Atividade 11}    &             &             &             &             &             &      X      &      X      &      X      &      X      &      X      &      X      \\
  \textbf{Atividade 12}     &             &             &             &             &             &             &             &             &             &             &      X      \\
  \bottomrule
\end{tabular}
\end{small}
\end{center}
\end{table}


\subsection{Organização do Documento}
%!TEX root = ../../main.tex

