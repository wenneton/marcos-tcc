A fundamentação teórica para a elaboração deste trabalho consiste em conceitos ligados ao \emph{Machine Learning}. Primeiramente, os conceitos gerais desta área serão apresentados na Seção \ref{subsec:ml}, seguidos pelas características das Redes Neurais Artificais na Seção \ref{subsec:rna}. As definições elementares da técnica de \emph{Machine Learning} conhecida como \emph{Deep Learning} são apresentadas na Seção \ref{subsec:dl}. A Seção \ref{subsubsec:cnns} discorre sobre as características das Redes Neurais Convolucionais. Por fim, nas Seções \ref{subsubsec:arq-cnns} e \ref{subsubsec:transfer}, são apresentadas algumas Arquiteturas Canônicas das Redes Neurais Convolucionais e a técnica de \emph{Transfer Learning}, conceitos emergentes envolvendo \emph{Deep Learning}.

%%%%%

\subsection{\emph{Machine Learning}}
\label{subsec:ml}
%%%%%

\subsection{Redes Neurais Artificiais}
\label{subsec:rna}

\subsubsection{\emph{Multilayer Perceptron}}
\label{subsubsec:mlp}

%%%%%

\subsection{\emph{Deep Learning}}
\label{subsec:dl}

\subsubsection{Redes Neurais Convolucionais}
\label{subsubsec:cnns}

\subsubsection{Arquiteturas canônicas de Redes Neurais Convolucionais}
\label{subsubsec:arq-cnns}

\subsubsection{\emph{Transfer Learning}}
\label{subsubsec:transfer}