A fundamentação teórica para a elaboração deste trabalho consiste em conceitos ligados ao \emph{Machine Learning}. Primeiramente, os conceitos gerais desta área serão apresentados na Seção \ref{subsec:ml}, seguidos pelas características das Redes Neurais Artificais na Seção \ref{subsec:rna}. As definições elementares da técnica de \emph{Machine Learning} conhecida como \emph{Deep Learning} são apresentadas na Seção \ref{subsec:dl}. A Seção \ref{subsubsec:cnns} discorre sobre as características das Redes Neurais Convolucionais. Por fim, nas Seções \ref{subsubsec:arq-cnns} e \ref{subsubsec:transfer}, são apresentadas algumas Arquiteturas Canônicas das Redes Neurais Convolucionais e a técnica de \emph{Transfer Learning}, sendo estes conceitos emergentes envolvendo \emph{Deep Learning}.

%%%%%

\subsection{\emph{Machine Learning}}
\label{subsec:ml}

%Procurar uma definição no Goodfellow ou Khan.
...De acordo com Murphy \cite{murphy} , \emph{Machine Learning} é definido como um conjunto de métodos que podem detectar padrões em dados automaticamente e, em seguida, utilizar os padrões detectados para predizer dados futuros, ou para realizar outros tipos de decisão sob algum tipo de incerteza.


%%%%%

\subsection{Redes Neurais Artificiais}
\label{subsec:rna}

\subsubsection{\emph{Multilayer Perceptron}}
\label{subsubsec:mlp}

%%%%%

\subsection{\emph{Deep Learning}}
\label{subsec:dl}

\subsubsection{Redes Neurais Convolucionais}
\label{subsubsec:cnns}

\subsubsection{Arquiteturas canônicas de Redes Neurais Convolucionais}
\label{subsubsec:arq-cnns}

\subsubsection{\emph{Transfer Learning}}
\label{subsubsec:transfer}