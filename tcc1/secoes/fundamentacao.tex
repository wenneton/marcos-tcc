A fundamentação teórica para a elaboração deste trabalho consiste em conceitos ligados ao \emph{Machine Learning}. Primeiramente, os conceitos gerais desta área serão apresentados na Seção \ref{subsec:ml}, seguidos pelas características das Redes Neurais Artificais na Seção \ref{subsec:rna}. As definições elementares da técnica de \emph{Machine Learning} conhecida como \emph{Deep Learning} são apresentadas na Seção \ref{subsec:dl}. A Seção \ref{subsubsec:cnns} discorre sobre as características das Redes Neurais Convolucionais, prosseguindo até a Seção \ref{subsubsec:arq-cnns} onde são apresentadas algumas de suas arquiteturas canônicas. Por fim, a Seção \ref{subsubsec:transfer} contém informações sobre a técnica conhecida como \emph{Transfer Learning}, um conceito emergente utilizado pelo \emph{Deep Learning}.

%%%%%

\subsection{\emph{Machine Learning}}
\label{subsec:ml}

%Procurar uma definição no Goodfellow ou Khan.
\emph{Machine Learning} (ML), também conhecido como Aprendizado de Máquina, é o estudo sistemático de algoritmos e sistemas que melhoram seu conhecimento ou desempenho com o uso da experiência \cite{flach}. Em 1959, o pioneiro em jogos de computador Arthur Samuels definiu \emph{Machine Learning} como um ``campo de estudos que dá aos computadores a habilidade de aprender sem serem explicitamente programados´´ \cite{simon}. De acordo com Murphy \cite{murphy} , \emph{Machine Learning} pode ser definido como um conjunto de métodos que podem detectar padrões em dados automaticamente e, em seguida, utilizar os padrões detectados para predizer dados futuros, ou para realizar outros tipos de decisão sob algum tipo de incerteza.

\emph{Machine Learning} é comumente dividido em três tipos principais de aprendizado, chamados de supervisionado, não-supervisionado e semi-supervisionado. No caso dos algoritmos de aprendizado supervisionado, o objetivo é aprender um mapeamento de entradas para saídas, dado um conjunto rotulado de pares de entradas e saídas. No aprendizado não supervisionado, o algoritmo é apresentado somente aos dados de entrada, e o seu propósito é encontrar padrões significativos nos mesmos. Este problema não é bem definido, porque geralmente não é especificado o tipo de padrão que deve ser encontrado nos dados. Além disso, diferentemente do aprendizado supervisionado, não existe nenhuma métrica de erro óbvia para utilizar. O aprendizado semi-supervisionado, por sua vez, normalmente combina uma pequena quantidade de dados rotulados com uma grande quantidade de dados não rotulados para criar um classificador próprio. Em alguns casos, a abordagem de aprendizagem semi-supervisionada pode ser de grande valor prático. \cite{khan}

Quanto aos tipos de saída dos modelos de ML, pode

% Citar os tipos de saída dos modelos de ML com Flach.
% Falar um pouco mais sobre classificação e regressão utilizando o Goodfellow.

%%%%%

\subsection{Redes Neurais Artificiais}
\label{subsec:rna}

\subsubsection{\emph{Multilayer Perceptron}}
\label{subsubsec:mlp}

%%%%%

\subsection{\emph{Deep Learning}}
\label{subsec:dl}

\subsubsection{Redes Neurais Convolucionais}
\label{subsubsec:cnns}

\subsubsection{Arquiteturas canônicas de Redes Neurais Convolucionais}
\label{subsubsec:arq-cnns}

\subsubsection{\emph{Transfer Learning}}
\label{subsubsec:transfer}