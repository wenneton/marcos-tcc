\todo{Apresentação da seção}


\subsection{Resultados Obtidos com a CNN LeNet}

% Visão geral de quantas CNNs deste tipo foram treinadas, houve uma busca em grid por todos os hiperparâmetros

% Considerando a métrica de F-Score como referência, foi possível identificar os melhores resultados para o conjunto de testes conforme TAbela X.

\begin{tabular}{cccccc}
\toprule
\textbf{Abordagem} & \textbf{Otimizador} & \textbf{\emph{Patience}}  & \textbf{Função de Ativação} & \textbf{Acurácia} & \textbf{F-Score} \\
\midrule
Abordagem 1 & \\
Abordagem 2 & \\
Abordagem 3 & \\
\bottomrule
\end{tabular}

Os gráficos da Figura X denotam o histórico da perda (\emph{loss}) e acurácia para o conjunto de treinamento e validação destas redes. Nota-se que nenhuma delas chegou ao limite máximo de épocas possíveis, interrompendo o aprendizado por meio de \emph{early stopping}.


%% Graficos aqui

Examinando mais atentamente o desempenho destas redes, tem-se, então, as matrizes de confusão mostradas na Figura Z.

%%%% ARGUMENTACAO



\subsection{Resultados Obtidos com a CNN AlexNet}
 %% Trabalhar aqui

 %% Quantitativo de CNNs treinadas

\subsection{Resultados Obtidos com a CNN SqueezeNet}

\subsection{Resultados Obtidos com a CNN MobileNet}
