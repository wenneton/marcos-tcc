%!TEX root = ../main.tex

\begin{frame}{Metodologia}
	\begin{itemize}
		\item[] A condução das atividades obedece à metodologia apresentada a seguir, composta dos seguintes passos:
		\medskip
		\begin{itemize}
			\footnotesize
			\item[1.] Estudo dos conceitos relacionados à Aprendizado de Máquina, Redes Neurais Convolucionais e \emph{Deep Learning};
			\medskip
			\medskip
			\item[2.] Descrição do problema considerado como uma tarefa de Aprendizado de Máquina;
			\medskip
			\medskip
			\item[3.] Consolidação de uma base de dados representativa de assinaturas originais e forjadas;
			\medskip
			\medskip
			\item[4.] Levantamento do ferramental tecnológico para implementação das redes neurais convolucionais;
			\medskip
			\medskip
			\item[5.] Proposição de modelos de redes neurais convolucionais para o problema considerado, contemplando arquitetura, parâmetros e hiperparâmetros;
		\end{itemize}
	\end{itemize}
\end{frame}

\begin{frame}{Metodologia}
\begin{itemize}
	\item[]
	\begin{itemize}
		\footnotesize
		\item[6.] Treino das redes propostas para a tarefa de aprendizado considerada;
		\medskip
		\medskip
		\item[7.] Teste das redes previamente treinadas com vistas a coleta de métricas de desempenho;
		\medskip
		\medskip
		\item[8.] Análise dos resultados e identificação dos modelos mais adequados para o problema considerado;
		\medskip
		\medskip
		\item[9.] Escrita da proposta de Trabalho de Conclusão de Curso;
		\medskip
		\medskip
		\item[10.] Defesa da proposta de Trabalho de Conclusão de Curso;
		\medskip
		\medskip
		\item[11.] Escrita do Trabalho de Conclusão de Curso; e
		\medskip
		\medskip
		\item[12.] Defesa do Trabalho de Conclusão de Curso.
	\end{itemize}
\end{itemize}
\end{frame}
