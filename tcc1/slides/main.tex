\documentclass[dvipsnames,table,mathserif,aspectratio=169]{beamer} % Definição do tipo de arquivo

\usepackage[T1]{fontenc}      %---------------------------------%
\usepackage[utf8]{inputenc} % Para língua portuguesa          %
\usepackage[brazil]{babel}    %---------------------------------%
\usepackage{times}
\usepackage[export]{adjustbox}
\usepackage{subcaption}

\usepackage[framesassubsections]{beamerprosper} % opções: email, institution,
\usepackage{caption}
\usepackage{stackengine}
\usepackage[portuguese,ruled,lined]{algorithm2e}
\usepackage{algorithmic}
\usepackage{graphicx,url}
\usepackage{enumerate}

\usepackage{amsmath,amsfonts,amsthm,bbm,amssymb}
\usepackage{epsfig}
\usepackage{booktabs}
\usepackage{scalefnt}
\usepackage{commath}

\usepackage{multirow}
% \usepackage[caption=true,font=footnotesize]{subfig}

% \makeatletter
% \patchcmd{\beamer@sectionintoc}
%   {\vfill}
%   {\vskip\itemsep}
%   {}
%   {}
% \makeatother

\captionsetup{compatibility=false}
\newtheorem{myobj}{Objetivos}
\newtheorem{myobjgeral}{Objetivo Geral}
\newtheorem{myobjesp}{Objetivos Específicos}
\usetheme{Execushares}
\newcolumntype{C}[1]{>{\centering\let\newline\\\arraybackslash\hspace{0pt}}m{#1}}


\title[TCC I]{\LARGE{\textbf{Verificação da Autenticidade de Assinaturas Manuscritas Utilizando Redes Neurais Convolucionais}}\\ \ \ \newline \small{Defesa do Trabalho de Conclusão de Curso I}}
\author[Araújo, Guedes]{\textbf{Marcos Wenneton V. de Araujo} \\ Orientadora: Elloá B. Guedes\\\small\email{\{mwvda.eng, ebgcosta\}@uea.edu.br} \\ }
\institute[LSI, EST, UEA]
{
  Grupo de Pesquisa em Sistemas Inteligentes\\
  Escola Superior de Tecnologia\\
  Universidade do Estado do Amazonas\\
  Manaus -- Amazonas -- Brasil
}
\date{\today}


\setcounter{showSlideNumbers}{1}

\AtBeginSection[]
{
  \begin{frame}<beamer>
    \frametitle{Agenda}
    \vskip2\baselineskip            %--------------------------------%
        \fontsize{10}{21}\selectfont% Modificar para caber de acordo %
        \tableofcontents[currentsection]% com a quantidade de seções  %
    \vskip0pt plus 1filll           %--------------------------------%
  \end{frame}
}

\begin{document}
\nocite{*}
\selectlanguage{brazil}

\maketitle

\begin{frame}{Agenda}
  \vskip2\baselineskip            %--------------------------------%
      \fontsize{10}{21}\selectfont% Modificar para caber de acordo %
      \tableofcontents            % com a quantidade de seções     %
  \vskip0pt plus 1filll           %--------------------------------%
\end{frame}

\section{Introdução}
%!TEX root = ../main.tex

\begin{frame}{Assinaturas manuscritas}
	\begin{itemize}
		\item Biometria
		\begin{itemize}
			\item Características fisiológicas
			\item Traços comportamentais
		\end{itemize}
		\bigskip
		\item Assinaturas manuscritas como forma de biometria
		\begin{itemize}
			\item Utilização desde os tempos primórdios
			\item Método não-invasivo
			\item Baixo custo de aquisição
		\end{itemize}
		\bigskip
		\item Difícil verificação de autenticidade devido a grande variabilidade dos padrões encontrados nas assinaturas
	\end{itemize}
\end{frame}


\section{Objetivos}
%!TEX root = ../main.tex
\begin{frame}{Objetivos}
	\begin{block}{Objetivo Geral}
	\emph{Verificar a autenticidade de assinaturas manuscritas utilizando Redes Neurais Convolucionais}
	\end{block}
	\pause
	\begin{block}{Objetivos Específicos}
		\begin{itemize}
		\footnotesize
		\item Realizar a fundamentação teórica acerca dos conceitos das redes neurais convolucionais;
		\item Consolidar uma base de dados representativa de assinaturas manuscritas;
		\item Descrever o problema considerado segundo uma tarefa de Aprendizado de Máquina;
		\item Propor, treinar e testar diferentes redes neurais convolucionais para a tarefa considerada;
		\item Analisar os resultados obtidos.
		\end{itemize}
	\end{block}
\end{frame}


\section{Justificativa}
%!TEX root = ../main.tex

\begin{frame}{Justificativa}
	\begin{itemize}
		\item Autenticação de assinaturas manuscritas
		\begin{itemize}
			\item Devido a ampla utilização em documentos oficiais e transações
			financeiras atualmente, busca-se a melhoria e avaliação de métodos para este fim;
			\item Documentos e obras de arte históricas.
		\end{itemize}
		\bigskip
		\item Prática de conceitos, técnicas e tecnologias de uma área emergente da Computação
		\bigskip
		\item Proposta alinhada com as atividades desenvolvidas pelo \alert{Laboratório de Sistemas Inteligentes}
	\end{itemize}
\end{frame}


\section{Metodologia}
%!TEX root = ../main.tex

\begin{frame}{Metodologia}
	\begin{itemize}
		\item[] A condução das atividades obedece à metodologia apresentada a seguir, composta dos seguintes passos:
		\medskip
		\begin{itemize}
			\footnotesize
			\item[1.] Estudo dos conceitos relacionados à Aprendizado de Máquina, Redes Neurais Convolucionais e \emph{Deep Learning};
			\medskip
			\medskip
			\item[2.] Descrição do problema considerado como uma tarefa de Aprendizado de Máquina;
			\medskip
			\medskip
			\item[3.] Consolidação de uma base de dados representativa de assinaturas originais e forjadas;
			\medskip
			\medskip
			\item[4.] Levantamento do ferramental tecnológico para implementação das redes neurais convolucionais;
			\medskip
			\medskip
			\item[5.] Proposição de modelos de redes neurais convolucionais para o problema considerado, contemplando arquitetura, parâmetros e hiperparâmetros;
		\end{itemize}
	\end{itemize}
\end{frame}

\begin{frame}{Metodologia}
\begin{itemize}
	\item[]
	\begin{itemize}
		\footnotesize
		\item[6.] Treino das redes propostas para a tarefa de aprendizado considerada;
		\medskip
		\medskip
		\item[7.] Teste das redes previamente treinadas com vistas a coleta de métricas de desempenho;
		\medskip
		\medskip
		\item[8.] Análise dos resultados e identificação dos modelos mais adequados para o problema considerado;
		\medskip
		\medskip
		\item[9.] Escrita da proposta de Trabalho de Conclusão de Curso;
		\medskip
		\medskip
		\item[10.] Defesa da proposta de Trabalho de Conclusão de Curso;
		\medskip
		\medskip
		\item[11.] Escrita do Trabalho de Conclusão de Curso; e
		\medskip
		\medskip
		\item[12.] Defesa do Trabalho de Conclusão de Curso.
	\end{itemize}
\end{itemize}
\end{frame}


\section{Cronograma}
%!TEX root = ../main.tex

\begin{frame}{Cronograma }
  \ \  \\[0.1cm]
  \begin{minipage}[b]{0.9\linewidth}
\begin{table}[h!]
\scalefont{0.5}
\begin{center}
\caption{Cronograma de atividades}
\label{tab:cronograma}
\begin{small}
\begin{tabular}{p{5cm}cccccccccccc}
  \toprule
  & &  &  & &  & \textbf{2019}  & &  &  &  &  & \\
                                        & \textbf{02} & \textbf{03} & \textbf{04} & \textbf{05} & \textbf{06} & \textbf{07} & \textbf{08} & \textbf{09} & \textbf{10} & \textbf{11} & \textbf{12} \\
  \midrule

  \textbf{Atividade 1}                     &      X      &      X      &      X      &           &             &             &             &             &             &             &             \\
  \textbf{Atividade 2} &             &      X      &            &             &             &             &             &             &             &             &             \\
  \textbf{Atividade 3}         &             &     X        &    X         &            &            &            &            &            &             &             &             \\
  \textbf{Atividade 4}         &             &             &    X         &            &            &            &            &            &             &             &             \\
  \textbf{Atividade 5}         &             &             &             &      X      &       X     &   X         &     X       &            &             &             &             \\
  \textbf{Atividade 6}         &             &             &             &      X      &     X       &     X       &    X        &            &             &             &             \\
  \textbf{Atividade 7}         &             &             &             &            &            &            &      X      &      X      &             &             &             \\
  \textbf{Atividade 8}         &             &             &             &            &            &            &           &            &   X          &      X       &             \\
  \textbf{Atividade 9}          &      X      &      X      &      X      &      X      &      X      &             &             &             &             &             &             \\

  \textbf{Atividade 10}          &             &             &             &             &      X      &             &             &             &             &             &             \\
  \textbf{Atividade 11}    &             &             &             &             &             &      X      &      X      &      X      &      X      &      X      &      X      \\
  \textbf{Atividade 12}     &             &             &             &             &             &             &             &             &             &             &      X      \\
  \bottomrule
\end{tabular}
\end{small}
\end{center}
\end{table}
\end{minipage}
\end{frame}


\section{Fundamentação Teórica}
%!TEX root = ../main.tex


\begin{frame}{Aprendizado de Máquina}
\begin{itemize}
	\item As técnicas de \alert{Aprendizado de Máquina} têm sido aplicadas com sucesso em um grande número de problemas reais em diversos domínios
	\bigskip
	\item Características: natureza inferencial e a boa capacidade de generalização dos métodos e técnicas desta área
	\bigskip
	\item Algoritmos capazes de aprender padrões por meio de exemplos, baseado-se em dados previamente disponíveis
	\bigskip
	\item Paradigmas de aprendizado supervisionado e não-supervisionado
\end{itemize}
\end{frame}

\begin{frame}{Redes Neurais Artificiais}
  \vskip2\baselineskip
	\begin{itemize}
		\item Inspiradas na capacidade de processamento de informações do cérebro humano
		\bigskip
		\item \alert{Neurônios artificiais} são as unidades fundamentais de uma RNA
		\item \alert{Função de ativação} fornece a resposta de um neurônio para uma dada entrada
		\bigskip
		\item Neurônios artificiais são conectados entre si na forma de uma rede e distribuídos em uma ou mais camadas ocultas
		\bigskip
		\item Algoritmo \emph{Backpropagation}
		\begin{itemize}
			\footnotesize
			\item Fase \emph{foward} -- produz uma saída para uma dada entrada
			\item Fase \emph{backwards} -- calcula a diferença entre as saídas para minimizar o erro
		\end{itemize}
	\end{itemize}
\end{frame}

\begin{frame}{\emph{Deep Learning} e Redes Neurais Convolucionais}
  \vskip2\baselineskip
\begin{itemize}
	\item Subárea específica do Aprendizado de Máquina
	\bigskip
	\item Redes Neurais Convolucionais (CNNs):
    \begin{itemize}
      \item Possuem camadas hierárquicas e profundas
      \item Aproveitam-se da operação matemática denominada convolução
      \item Destacam-se pelo reconhecimento de padrões em dados de alta dimensionalidade
    \end{itemize}

    \begin{figure}
    	\caption{Papel das camadas convolucionais e \emph{feature maps} das CNNs}
    	\label{fig:camadas-convolucionais}
    	\includegraphics[width=0.5\textwidth]{./img/camadas-convolucionais}
    \end{figure}

\end{itemize}
\end{frame}


\begin{frame}{\Large{Arquiteturas Canônicas de Redes Neurais Convolucionais}}
\begin{itemize}
	\item Arquiteturas com bom desempenho em competições de Visão Computacional
	\item Comuns ainda hoje no cenário de \emph{Deep Learning}
	\bigskip
	\item LeNet (1998)
	\item AlexNet (2012)
	\item VGG (2014)
	\item Inception (2014)
	\item ResNet (2015)
\end{itemize}

\end{frame}


\section{Solução Proposta}


\section{Visão Geral do Conjunto de Dados}
% estatisticas calculadas lá no notebook



\section{Preparação do Conjunto de Dados}

\begin{table}[h!]
	\centering
	\caption{Divisão dos dados}
	\label{tab:divisao-dados}
	%\scalefont{0.77}
	\begin{tabular}{c c c c c}
		\toprule
		\textbf{Abordagem} & \textbf{Tipo de Exemplo} & \textbf{Treino} & \textbf{Validação} & \textbf{Teste}\\
		\midrule
		\multirow{2}{*}{1} & genuíno & 2011 & 299 & 618 \\
     & forjado & 11649 & 1648 & 3237 \\
     \midrule
    \multirow{2}{*}{2} & genuíno & 2011 & 299 & 618   \\
     & forjado & 2024 & 308 & 569 \\
		\bottomrule
	\end{tabular}
\end{table}

\begin{figure}[h!]
  \centering
\caption{Visualização da divisão dos dados}
  \subfloat[Abordagem 1\label{subfig:approach1}]{%
    \includegraphics[width=0.7\textwidth]{imgs/approach1}
  }
  \hfill
  \subfloat[Abordagem 2\label{subfig:approach2}]{%
    \includegraphics[width=0.7\textwidth]{imgs/approach2}
  }
  \label{fig:divisao-dados}
\end{figure}


\section{Resultados Parciais}
%!TEX root = ../main.tex

\begin{frame}{Resultados Finais}
 \begin{itemize}
   \item Utilização de um servidor para treinamento das CNNs:
    \begin{itemize}
      \item Processador Intel Core i7
      \item 16 GB de RAM
      \item GPU Nvidia GeForce GTX 1080 com 11 GB de memória
    \end{itemize}
    \bigskip
    \item Modelos degenerados tiveram seus resultados descartados
    \begin{itemize}
      \item \emph{Dying ReLU problem}
      \item Permanência em mínimos locais no treinamento
    \end{itemize}
   \end{itemize}
\end{frame}

\begin{frame}{LeNet}

  \begin{table}[h]
    \centering
    \caption{Detalhamento dos melhores resultados obtidos com a arquitetura LeNet.}
    \label{tab:lenet}
    \resizebox{\textwidth}{!}{\begin{tabular}{ccccccc}
    \toprule
    \textbf{Abordagem} & \textbf{Otimizador} & \textbf{\emph{Patience}}  & \textbf{Função de Ativação} & \textbf{Acurácia} & \textbf{\emph{F-Score}} & \textbf{EER} \\
    \midrule
    Abordagem A & RMSprop & 5 & ReLU & $0.9865$ & $0.9755$ & $1.1679$\\
    Abordagem B & Adam & 10 & ELU & $0.8361$ & $0.8159$ & $12.5245$ \\
    \bottomrule
    \end{tabular}}
    \end{table}

\end{frame}

\begin{frame}{LeNet}

  \begin{figure}[h!]
    \centering
    \caption{Histórico de \emph{loss} e acurácia durante o treinamento dos melhores modelos obtidos com a arquitetura LeNet.}
    \begin{subfigure}{0.3\linewidth}
      \caption{\emph{Loss} LeNet A.\label{subfig:lenet-a-loss}}
      \includegraphics[width=\linewidth]{img/lenet-a-loss}%
    \end{subfigure}
    \hspace{1.5cm}
    \begin{subfigure}{0.3\linewidth}
      \caption{Acurácia LeNet A.\label{subfig:lenet-a-acc}}
      \includegraphics[width=\linewidth]{img/lenet-a-acc}%
    \end{subfigure}
    \hspace{1.5cm}
    \begin{subfigure}{0.3\linewidth}
      \caption{\emph{Loss} LeNet B.\label{subfig:lenet-b-loss}}
      \includegraphics[width=\linewidth]{img/lenet-b-loss}%
    \end{subfigure}
    \hspace{1.5cm}
    \begin{subfigure}{0.3\linewidth}
      \caption{Acurácia LeNet B.\label{subfig:lenet-b-acc}}
      \includegraphics[width=\linewidth]{img/lenet-b-acc}%
    \end{subfigure}
    \label{fig:treinamento-lenet}
  \end{figure}

\end{frame}

\begin{frame}{LeNet}
  \vskip1\baselineskip
  \begin{figure}[ht!]
      \caption{Matrizes de confusão dos melhores modelos obtidos com a arquitetura LeNet.}\label{fig:matrizes-lenet}
      \begin{subfigure}{0.4\linewidth}
        \caption{LeNet A}
        \includegraphics[width=\linewidth]{img/matriz-lenet-a}
      \end{subfigure}
      \hspace{2cm}
      \begin{subfigure}{0.4\linewidth}
        \caption{LeNet B}
        \includegraphics[width=\linewidth]{img/matriz-lenet-b}%
      \end{subfigure}
  \end{figure}
\end{frame}


\begin{frame}{AlexNet}

  \begin{table}[h!]
    \centering
    \caption{Detalhamento dos melhores modelos obtidos com a arquitetura AlexNet para cada uma das abordagens consideradas neste trabalho.}
    \label{tab:alexnet}
    \resizebox{\textwidth}{!}{\begin{tabular}{ccccccc}
    \toprule
    \textbf{Abordagem} & \textbf{Otimizador} & \textbf{\emph{Patience}}  & \textbf{Função de Ativação} & \textbf{Acurácia} & \textbf{F-Score} & \textbf{EER} \\
    \midrule
    Abordagem A & Adam & 15 & ELU & $0.9654$ & $0.9393$ & $1.5401$\\
    Abordagem B & RMSprop & 5 & ELU & $0.8593$ & $0.7993$ & $13.8265$\\
    \bottomrule
    \end{tabular}}
    \end{table}

\end{frame}

\begin{frame}{AlexNet}

  \begin{figure}[h!]
    \centering
    \caption{Histórico de \emph{loss} e acurácia durante o treinamento dos melhores modelos obtidos com a arquitetura AlexNet.}
    \begin{subfigure}{0.3\linewidth}
      \caption{\emph{Loss} AlexNet A.\label{subfig:alexnet-a-loss}}
      \includegraphics[width=\linewidth]{img/alexnet-a-loss}%
    \end{subfigure}
    \hspace{1.5cm}
    \begin{subfigure}{0.3\linewidth}
      \caption{Acurácia AlexNet A.\label{subfig:alexnet-a-acc}}
      \includegraphics[width=\linewidth]{img/alexnet-a-acc}%
    \end{subfigure}
    \hspace{1.5cm}
    \begin{subfigure}{0.3\linewidth}
      \caption{\emph{Loss} AlexNet B.\label{subfig:alexnet-b-loss}}
      \includegraphics[width=\linewidth]{img/alexnet-b-loss}%
    \end{subfigure}
    \hspace{1.5cm}
    \begin{subfigure}{0.3\linewidth}
      \caption{Acurácia AlexNet B.\label{subfig:alexnet-b-acc}}
      \includegraphics[width=\linewidth]{img/alexnet-b-acc}%
    \end{subfigure}
    \label{fig:treinamento-alexnet}
  \end{figure}
\end{frame}

\begin{frame}{AlexNet}
  \vskip1\baselineskip
  \begin{figure}[ht!]
    \caption{Matrizes de confusão dos melhores modelos obtidos com a arquitetura AlexNet.}\label{fig:matrizes-lenet}
    \begin{subfigure}{0.4\linewidth}
      \caption{AlexNet A}
      \includegraphics[width=\linewidth]{img/matriz-alexnet-a}
    \end{subfigure}
    \hspace{2cm}
    \begin{subfigure}{0.4\linewidth}
      \caption{AlexNet B}
      \includegraphics[width=\linewidth]{img/matriz-alexnet-b}%
    \end{subfigure}
\end{figure}
\end{frame}

\begin{frame}{MobileNet}

  \begin{table}[h!]
    \centering
    \caption{Detalhamento dos melhores modelos obtidos com a arquitetura MobileNet para cada uma das abordagens consideradas neste trabalho.}
    \label{tab:mobilenet}
    \resizebox{\textwidth}{!}{\begin{tabular}{ccccccc}
    \toprule
    \textbf{Abordagem} & \textbf{Otimizador} & \textbf{\emph{Patience}}  & \textbf{Função de Ativação} & \textbf{Acurácia} & \textbf{F-Score} & \textbf{EER} \\
    \midrule
    Abordagem A & SGD & 15 & ReLU & $0.9606$ & $0.9318$ & $0.9304$ \\
    Abordagem B & Adam & 15 & ReLU & $0.8856$ & $0.8658$ & $9.9475$\\
    \bottomrule
    \end{tabular}}
    \end{table}

\end{frame}

\begin{frame}{MobileNet}

  \begin{figure}[h!]
    \centering
    \caption{Histórico de \emph{loss} e acurácia durante o treinamento dos melhores modelos obtidos com a arquitetura MobileNet.}
    \begin{subfigure}{0.3\linewidth}
      \caption{\emph{Loss} MobileNet A.\label{subfig:alexnet-a-loss}}
      \includegraphics[width=\linewidth]{img/mobilenet-a-loss}%
    \end{subfigure}
    \hspace{1.5cm}
    \begin{subfigure}{0.3\linewidth}
      \caption{Acurácia MobileNet A.\label{subfig:alexnet-a-acc}}
      \includegraphics[width=\linewidth]{img/mobilenet-a-acc}%
    \end{subfigure}
    \hspace{1.5cm}
    \begin{subfigure}{0.3\linewidth}
      \caption{\emph{Loss} MobileNet B.\label{subfig:alexnet-b-loss}}
      \includegraphics[width=\linewidth]{img/mobilenet-b-loss}%
    \end{subfigure}
    \hspace{1.5cm}
    \begin{subfigure}{0.3\linewidth}
      \caption{Acurácia MobileNet B.\label{subfig:alexnet-b-acc}}
      \includegraphics[width=\linewidth]{img/mobilenet-b-acc}%
    \end{subfigure}
    \label{fig:treinamento-alexnet}
  \end{figure}
\end{frame}

\begin{frame}{MobileNet}
  \vskip1\baselineskip
  \begin{figure}[ht!]
    \caption{Matrizes de confusão dos melhores modelos obtidos com a arquitetura MobileNet.}\label{fig:matrizes-lenet}
    \begin{subfigure}{0.4\linewidth}
      \caption{MobileNet A}
      \includegraphics[width=\linewidth]{img/matriz-mobilenet-a}
    \end{subfigure}
    \hspace{2cm}
    \begin{subfigure}{0.4\linewidth}
      \caption{MobileNet B}
      \includegraphics[width=\linewidth]{img/matriz-mobilenet-b}%
    \end{subfigure}
\end{figure}
\end{frame}

\begin{frame}{ShuffleNet}

  \begin{table}[h!]
    \centering
    \caption{Detalhamento dos modelos obtidos com a arquitetura ShuffleNet para cada uma das abordagens consideradas.}
    \label{tab:shufflenet}
    \resizebox{\textwidth}{!}{\begin{tabular}{ccccccc}
    \toprule
    \textbf{Abordagem} & \textbf{Otimizador} & \textbf{\emph{Patience}}  & \textbf{Função de Ativação} & \textbf{Acurácia} & \textbf{F-Score} & \textbf{EER} \\
    \midrule
    Abordagem A & RMSprop & 15 & ReLU & $0.9404$ & $0.9004$ & $7.5400$ \\
    Abordagem B & RMSprop & 15 & ReLU & $0.8345$ & $0.7705$ & $23.8151$\\
    \bottomrule
    \end{tabular}}
    \end{table}

\end{frame}

\begin{frame}{ShuffleNet}

  \begin{figure}[h!]
    \centering
    \caption{Histórico de \emph{loss} e acurácia durante o treinamento dos modelos obtidos com a arquitetura ShuffleNet.}
    \begin{subfigure}{0.3\linewidth}
      \caption{\emph{Loss} ShuffleNet A.\label{subfig:alexnet-a-loss}}
      \includegraphics[width=\linewidth]{img/shufflenet-a-loss}%
    \end{subfigure}
    \hspace{1.5cm}
    \begin{subfigure}{0.3\linewidth}
      \caption{Acurácia ShuffleNet A.\label{subfig:alexnet-a-acc}}
      \includegraphics[width=\linewidth]{img/shufflenet-a-acc}%
    \end{subfigure}
    \hspace{1.5cm}
    \begin{subfigure}{0.3\linewidth}
      \caption{\emph{Loss} ShuffleNet B.\label{subfig:alexnet-b-loss}}
      \includegraphics[width=\linewidth]{img/shufflenet-b-loss}%
    \end{subfigure}
    \hspace{1.5cm}
    \begin{subfigure}{0.3\linewidth}
      \caption{Acurácia ShuffleNet B.\label{subfig:alexnet-b-acc}}
      \includegraphics[width=\linewidth]{img/shufflenet-b-acc}%
    \end{subfigure}
    \label{fig:treinamento-alexnet}
  \end{figure}
\end{frame}

\begin{frame}{ShuffleNet}
  \vskip1\baselineskip
  \begin{figure}[ht!]
    \caption{Matrizes de confusão dos modelos obtidos com a arquitetura ShuffleNet.}\label{fig:matrizes-lenet}
    \begin{subfigure}{0.4\linewidth}
      \caption{ShuffleNet A}
      \includegraphics[width=\linewidth]{img/matriz-shufflenet-a}
    \end{subfigure}
    \hspace{2cm}
    \begin{subfigure}{0.4\linewidth}
      \caption{ShuffleNet B}
      \includegraphics[width=\linewidth]{img/matriz-shufflenet-b}%
    \end{subfigure}
\end{figure}
\end{frame}

\begin{frame}{SqueezeNet}

  \begin{table}[h!]
    \centering
    \caption{Detalhamento dos modelos obtidos com a arquitetura SqueezeNet para cada uma das abordagens consideradas neste trabalho.}
    \label{tab:squeezenet}
    \resizebox{\textwidth}{!}{\begin{tabular}{ccccccc}
    \toprule
    \textbf{Abordagem} & \textbf{Otimizador} & \textbf{\emph{Patience}}  & \textbf{Função de Ativação} & \textbf{Acurácia} & \textbf{F-Score} & \textbf{EER} \\
    \midrule
    Abordagem A & RMSprop & 15 & ReLU & $0.9048$ & $0.8948$ & $11.5074$ \\
    Abordagem B & RMSprop & 15 & ReLU & $0.8210$ & $0.7709$ & $20.1673$\\
    \bottomrule
    \end{tabular}}
    \end{table}

\end{frame}

\begin{frame}{SqueezeNet}

  \begin{figure}[h!]
    \centering
    \caption{Histórico de \emph{loss} e acurácia durante o treinamento dos modelos obtidos com a arquitetura SqueezeNet.}
    \begin{subfigure}{0.3\linewidth}
      \caption{\emph{Loss} SqueezeNet A.\label{subfig:squeezenet-a-loss}}
      \includegraphics[width=\linewidth]{img/squeezenet-a-loss}%
    \end{subfigure}
    \hspace{1.5cm}
    \begin{subfigure}{0.3\linewidth}
      \caption{Acurácia SqueezeNet A.\label{subfig:squeezenet-a-acc}}
      \includegraphics[width=\linewidth]{img/squeezenet-a-acc}%
    \end{subfigure}
    \hspace{1.5cm}
    \begin{subfigure}{0.3\linewidth}
      \caption{\emph{Loss} SqueezeNet B.\label{subfig:squeezenet-b-loss}}
      \includegraphics[width=\linewidth]{img/squeezenet-b-loss}%
    \end{subfigure}
    \hspace{1.5cm}
    \begin{subfigure}{0.3\linewidth}
      \caption{Acurácia SqueezeNet B.\label{subfig:squeezenet-b-acc}}
      \includegraphics[width=\linewidth]{img/squeezenet-b-acc}%
    \end{subfigure}
    \label{fig:treinamento-alexnet}
  \end{figure}
\end{frame}

\begin{frame}{SqueezeNet}
  \vskip1\baselineskip
  \begin{figure}[ht!]
    \caption{Matrizes de confusão dos modelos obtidos com a arquitetura SqueezeNet.}\label{fig:matrizes-lenet}
    \begin{subfigure}{0.4\linewidth}
      \caption{SqueezeNet A}
      \includegraphics[width=\linewidth]{img/matriz-squeezenet-a}
    \end{subfigure}
    \hspace{2cm}
    \begin{subfigure}{0.4\linewidth}
      \caption{SqueezeNet B}
      \includegraphics[width=\linewidth]{img/matriz-squeezenet-b}%
    \end{subfigure}
\end{figure}
\end{frame}

\begin{frame}{VGG-16}

  \begin{itemize}
    \item Treinada apenas para abordagem B, com hiperparâmetros \emph{Ad Hoc}
  \end{itemize}

  \begin{table}[h!]
    \centering
    \caption{Detalhamento do modelo obtido com a arquitetura VGG-16 para a abordagem B.}
    \label{tab:vgg}
    \begin{tabular}{cccccc}
    \toprule
    \textbf{Otimizador} & \textbf{\emph{Patience}}  & \textbf{Função de Ativação} & \textbf{Acurácia} & \textbf{F-Score} & \textbf{EER} \\
    \midrule
    RMSprop & 10 & ELU & $0.8391$ & $0.8019$ & $16.1096$ \\
    \bottomrule
    \end{tabular}
    \end{table}

\end{frame}

\begin{frame}{VGG-16}

  \begin{figure}[h!]
    \centering
    \caption{Histórico de \emph{loss} e acurácia durante o treinamento do modelo obtido com a arquitetura VGG-16.}
    \begin{subfigure}{0.44\linewidth}
      \caption{\emph{Loss} VGG-16 B.\label{subfig:squeezenet-b-loss}}
      \includegraphics[width=\linewidth]{img/vgg-b-loss}%
    \end{subfigure}
    \hspace{1.5cm}
    \begin{subfigure}{0.44\linewidth}
      \caption{Acurácia VGG-16 B.\label{subfig:squeezenet-b-acc}}
      \includegraphics[width=\linewidth]{img/vgg-b-acc}%
    \end{subfigure}
    \label{fig:treinamento-alexnet}
  \end{figure}
\end{frame}

\begin{frame}{VGG-16}
  \vskip1\baselineskip
  \begin{figure}[h]
    \centering
    \caption{Matriz de confusão do modelo obtido com a arquitetura VGG-16.}\label{fig:matrizes-vgg}
    \includegraphics[width=0.6\textwidth]{img/matriz-vgg}
\end{figure}
\end{frame}

\begin{frame}{InceptionV3}

  \begin{itemize}
    \item Treinada apenas para abordagem B, com hiperparâmetros \emph{Ad Hoc}
  \end{itemize}

  \begin{table}[h!]
    \centering
    \caption{Detalhamento do modelo obtido com a arquitetura Inception-V3 para a abordagem B.}
    \label{tab:inception}
    \begin{tabular}{cccccc}
    \toprule
    \textbf{Otimizador} & \textbf{\emph{Patience}}  & \textbf{Função de Ativação} & \textbf{Acurácia} & \textbf{F-Score} & \textbf{EER} \\
    \midrule
    RMSprop & 5 & ELU & $0.8394$ & $0.8070$ & $16.9493$ \\
    \bottomrule
    \end{tabular}
    \end{table}

\end{frame}

\begin{frame}{InceptionV3}

  \begin{figure}[h!]
    \centering
    \caption{Histórico de \emph{loss} e acurácia durante o treinamento do modelo obtido com a arquitetura InceptionV3.}
    \begin{subfigure}{0.44\linewidth}
      \caption{\emph{Loss} InceptionV3 B.\label{subfig:squeezenet-b-loss}}
      \includegraphics[width=\linewidth]{img/inception-b-loss}%
    \end{subfigure}
    \hspace{1.5cm}
    \begin{subfigure}{0.44\linewidth}
      \caption{Acurácia InceptionV3 B.\label{subfig:squeezenet-b-acc}}
      \includegraphics[width=\linewidth]{img/inception-b-acc}%
    \end{subfigure}
    \label{fig:treinamento-alexnet}
  \end{figure}
\end{frame}

\begin{frame}{InceptionV3}
  \vskip1\baselineskip
  \begin{figure}[h]
    \centering
    \caption{Matriz de confusão do modelo obtido com a arquitetura InceptionV3.}\label{fig:matrizes-vgg}
    \includegraphics[width=0.6\textwidth]{img/matriz-inception}
\end{figure}
\end{frame}

\section{Considerações Parciais}
%!TEX root = ../main.tex

\begin{frame}{Considerações Finais}
  \vskip1\baselineskip
  \begin{itemize}
    \item $222$ redes foram treinadas e testadas com um total de $27.962$ exemplos
    \pause
    \bigskip
    \item Melhor desempenho Abordagem A: LeNet
    \begin{itemize}
      \item \alert{Parâmetros e Hiperparâmetros}: Otimizador RMSprop, \emph{patience} 5 e função de ativação \emph{Leaky} ReLU.
      \item \alert{Acurácia}: $0.9865$
      \item \alert{\emph{F-Score}}: $0.9755$
      \item \alert{EER}: $1.17\%$
    \end{itemize}
    \pause
    \item Melhor desempenho Abordagem B: MobileNet
    \begin{itemize}
      \item \alert{Parâmetros e Hiperparâmetros}: Otimizador Adam, \emph{patience} 15 e função de ativação ReLU.
      \item \alert{Acurácia}: $0.8856$
      \item \alert{\emph{F-Score}}: $0.8658$
      \item \alert{EER}: $9.94\%$
    \end{itemize}
  \end{itemize}
\end{frame}

\begin{frame}{Considerações Finais}
  \begin{itemize}
    \item Trabalhos futuros:
    \begin{itemize}
      \item Encontrar modelos mais compactos
      \item Remoção de mapas de calor afim de ajudar na checagem de assinaturas por revisores humanos
    \end{itemize}
  \end{itemize}
\end{frame}


\section{Referências}
%!TEX root = ../main.tex

% \begin{frame}{Referências}
% \begin{itemize}
% \footnotesize{
% \item BRAGA, A. de P.; CARVALHO, A. P. de Leon F. de; LUDERMIR, T.B. \emph{Redes Neurais Artificiais: Teorias e Aplicações}. Rio de Janeiro, RJ: Livros Técnicos e Científicos Editora S.A., 2000.
% \ \ \newline
% \item BLANKERS, V. L. et al. \emph{The icdar 2009 signature verification competition}. In: \emph{10th International Conference on Document Analysis and Recognition}. Barcelona, Catalonia, Spain: IEEE, 2009. p. 1403-1407.
% \ \ \newline
% \item KHAN, S. et. al. \emph{A Guide to Convolutional Neural Networks for Computer Vision}. Austrália: Morgan \& Claypool, 2018.
% \ \ \newline
% \item LIWICKI, M. \emph{IAPR TC11 - ICDAR 2009 Signature Verification Competition (SigComp2009)}. 2012. Disponível em: hhttp://www.iapr-tc11.org/mediawiki/index.php?title=IAPR-TC11:Reading Systemsi. Acesso em 5 de março de 2019.
% \end{itemize}
% \end{frame}


\section*{}
\maketitle

\end{document}
