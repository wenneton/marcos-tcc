\documentclass[12pt]{article}

\usepackage{sbc-template}

\usepackage{booktabs}
\usepackage{graphicx}
\usepackage{float}
\usepackage{url}
\usepackage{hyperref}
\usepackage{enumerate}
\usepackage{amsmath,bbm}

\usepackage{siunitx}
\sisetup{output-decimal-marker = {,}}

\usepackage[utf8]{inputenc}
\usepackage[brazil]{babel}

\usepackage[alf]{abntex2cite}

\usepackage{multirow}
\usepackage{multicol}
\usepackage{array}
\newcolumntype{C}[1]{>{\centering\let\newline\\\arraybackslash\hspace{0pt}}m{#1}}
\usepackage{scalefnt}

%% DEFINIÇÃO DE QUADRO
\usepackage{mdframed}
\usepackage{trivfloat}
\usepackage{blindtext}
\trivfloat{quadro}
\floatstyle{plaintop} %Força posição da legenda para o topo
\restylefloat{quadro} %Força posição da legenda para o topo
\renewcommand{\quadroname}{Quadro}
\renewcommand{\listquadroname}{Lista de quadros} %Forçar texto na lista de quadros

%%%%%%%%%%%%%% Python %%%%%%%%%%%%
\usepackage{textcomp}
\usepackage{listings}
\renewcommand{\lstlistlistingname}{Code Listings}
\renewcommand{\lstlistingname}{Code Listing}
\definecolor{gray}{gray}{0.5}
\definecolor{key}{rgb}{0,0.5,0}
\lstnewenvironment{python}[1][]{
\lstset{
language=python,
basicstyle=\ttfamily\small,
otherkeywords={1, 2, 3, 4, 5, 6, 7, 8 ,9 , 0, -, =, +, [, ], (, ), \{, \}, :, *, !},
keywordstyle=\color{blue},
stringstyle=\color{red},
showstringspaces=false,
emph={class, pass, in, for, while, if, is, elif, else, not, and, or,
def, print, exec, break, continue, return},
emphstyle=\color{black}\bfseries,
emph={[2]True, False, None, self},
emphstyle=[2]\color{key},
emph={[3]from, import, as},
emphstyle=[3]\color{blue},
upquote=true,
morecomment=[s]{"""}{"""},
commentstyle=\color{gray}\slshape,
framexleftmargin=0mm, framextopmargin=0mm,#1
}}{} 

%%%%%%%%%%%%%%%%%%%%%%%%%%%%%%%%%%

%% Comentários
\usepackage{xcolor}
\definecolor{lightblue}{RGB}{0,191,255}
\setlength{\marginparwidth}{2cm}
\usepackage[textsize=tiny,backgroundcolor=lightblue,linecolor=lightblue]{todonotes}


%%%%%% Subfiguras -- elloa
\usepackage[bf,sf,footnotesize,indent,justification=centering]{caption}
\usepackage[caption=true,font=footnotesize]{subfig}
\captionsetup[subfigure]{justification=centering,labelfont={bf,sf},textfont={bf,sf,footnotesize},singlelinecheck=off,justification=centering}
\captionsetup[figure]{justification=centering,labelfont={bf,sf},textfont={bf,sf,footnotesize},singlelinecheck=off}
\captionsetup[table]{justification=centering,labelfont={bf,sf},textfont={bf,sf,footnotesize},singlelinecheck=off,justification=centering}
%%%%%%%%%%%%%%%%%%%%%%%%%%%%%%%%%%%%%%%%%%


\sloppy

\title{Verificação de Autenticidade de Assinaturas Manuscritas\\ Utilizando Redes Neurais Convolucionais}

\author{Marcos Wenneton Vieira de Araújo\\
Orientadora: Elloá B. Guedes}

\address{Laboratório de Sistemas Inteligentes\\Grupo de Pesquisas em Sistemas Inteligentes\\
Escola Superior de Tecnologia\\
Universidade do Estado do Amazonas\\
Av. Darcy Vargas, 1200, Manaus, AM
  \email{\{mwvda.eng, ebgcosta\}@uea.edu.br}
}



\begin{document}

\maketitle
\pagestyle{plain} %% Não remover, introduz numeração de páginas -- elloa

\section{Introdução} \label{sec:introducao}
%!TEX root = ../main.tex

\begin{frame}{Assinaturas manuscritas}
	\begin{itemize}
		\item Biometria
		\begin{itemize}
			\item Características fisiológicas
			\item Traços comportamentais
		\end{itemize}
		\bigskip
		\item Assinaturas manuscritas como forma de biometria
		\begin{itemize}
			\item Utilização desde os tempos primórdios
			\item Método não-invasivo
			\item Baixo custo de aquisição
		\end{itemize}
		\bigskip
		\item Difícil verificação de autenticidade devido a grande variabilidade dos padrões encontrados nas assinaturas
	\end{itemize}
\end{frame}


\section{Fundamentação Teórica} \label{sec:fund-teor}
%%%%%

\subsection{\emph{Machine Learning}}

%%%%%

\subsection{Redes Neurais Artificiais}

\subsubsection{\emph{Multilayer Perceptron}}

%%%%%

\subsection{\emph{Deep Learning}}

\subsubsection{Redes Neurais Convolucionais}

\subsubsection{Arquiteturas canônicas de Redes Neurais Convolucionais}

\subsubsection{\emph{Transfer Learning}}

\section{Trabalhos Relacionados} \label{sec:trab-rel}

\section{Solução Proposta} \label{sec:sol-prop}


\section{Visão Geral do Conjunto de Dados}
% estatisticas calculadas lá no notebook



\section{Preparação do Conjunto de Dados}

\begin{table}[h!]
	\centering
	\caption{Divisão dos dados}
	\label{tab:divisao-dados}
	%\scalefont{0.77}
	\begin{tabular}{c c c c c}
		\toprule
		\textbf{Abordagem} & \textbf{Tipo de Exemplo} & \textbf{Treino} & \textbf{Validação} & \textbf{Teste}\\
		\midrule
		\multirow{2}{*}{1} & genuíno & 2011 & 299 & 618 \\
     & forjado & 11649 & 1648 & 3237 \\
     \midrule
    \multirow{2}{*}{2} & genuíno & 2011 & 299 & 618   \\
     & forjado & 2024 & 308 & 569 \\
		\bottomrule
	\end{tabular}
\end{table}

\begin{figure}[h!]
  \centering
\caption{Visualização da divisão dos dados}
  \subfloat[Abordagem 1\label{subfig:approach1}]{%
    \includegraphics[width=0.7\textwidth]{imgs/approach1}
  }
  \hfill
  \subfloat[Abordagem 2\label{subfig:approach2}]{%
    \includegraphics[width=0.7\textwidth]{imgs/approach2}
  }
  \label{fig:divisao-dados}
\end{figure}


\section{Resultados Parciais} \label{sec:res-parc}


\section{Considerações Parciais} \label{sec:cons-parc}





\bibliography{sbc-template}

\end{document}
